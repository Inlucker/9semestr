\documentclass[a4paper,oneside,14pt]{extreport}

\usepackage[T2A]{fontenc}
\usepackage[utf8]{inputenc}
\usepackage[english,russian]{babel}

%\usepackage[left=30mm, right=20mm, top=20mm, bottom=20mm]{geometry}
\usepackage[left=20mm, right=10mm, top=5mm, bottom=20mm]{geometry}

\usepackage{microtype}
\sloppy

\usepackage{setspace}
\onehalfspacing

\usepackage{indentfirst}
\setlength{\parindent}{12.5mm}

\usepackage{titlesec}
\titleformat{\chapter}{\LARGE\bfseries}{\thechapter}{14pt}{\LARGE\bfseries}
\titlespacing*{\chapter}{\parindent}{0mm}{5mm}
\titleformat{\section}{\Large\bfseries}{\thesection}{14pt}{\Large\bfseries}

\addto{\captionsrussian}{\renewcommand*{\contentsname}{Содержание}}
\usepackage{natbib}
\renewcommand{\bibsection}{\chapter*{Список использованных источников}}

\usepackage{caption}

\usepackage{wrapfig}
\usepackage{float}

\usepackage{graphicx}
\newcommand{\imgwc}[4]
{
	\begin{figure}[#1]
		\center{\includegraphics[width=#2]{inc/img/#3}}
		\caption{#4}
		\label{img:#3}
	\end{figure}
}
\newcommand{\imghc}[4]
{
	\begin{figure}[#1]
		\center{\includegraphics[height=#2]{inc/img/#3}}
		\caption{#4}
		\label{img:#3}
	\end{figure}
}
\newcommand{\imgsc}[4]
{
	\begin{figure}[#1]
		\center{\includegraphics[scale=#2]{inc/img/#3}}
		\caption{#4}
		\label{img:#3}
	\end{figure}
}

\usepackage{pgfplots}
\pgfplotsset{compat=newest}

\usepackage{listings}
\usepackage{listingsutf8}
\lstset{
	basicstyle=\footnotesize\ttfamily,
	keywordstyle=\color{blue},
	stringstyle=\color{red},
	commentstyle=\color{gray},
	numbers=left,
	numberstyle=\tiny,
	numbersep=5pt,
	frame=false,
	breaklines=true,
	breakatwhitespace=true,
	inputencoding=utf8/koi8-r
}

\lstdefinestyle{c}{
	language=C++,
	backgroundcolor=\color{white},
	basicstyle=\footnotesize\ttfamily,
	keywordstyle=\color{blue},
	stringstyle=\color{red},
	commentstyle=\color{gray},
	directivestyle=\color{orange},
	numbers=left,
	numberstyle=\tiny,
	stepnumber=1,
	numbersep=5pt,
	frame=single,
	tabsize=4,
	captionpos=t,
	breaklines=true,
	breakatwhitespace=true,
	escapeinside={\#*}{*)},
	morecomment=[l][\color{magenta}]{\#},
	columns=fullflexible
}

\newcommand{\code}[1]{\texttt{#1}}

\usepackage{amsmath}
\usepackage{amssymb}

\usepackage[unicode]{hyperref}
\hypersetup{hidelinks}

\makeatletter
\newcommand{\vhrulefill}[1]
{
	\leavevmode\leaders\hrule\@height#1\hfill \kern\z@
}
\makeatother

\begin{document}

\chapter*{Домашнее задание}

Задание:
\begin{enumerate}
	\item Реализовать один из алгоритмов сортировки (сортировка Хоара) последовательно и параллельно, с использованием технологии OpenMP на языках C и Python.
	\item Провести исследование реализованных алгоритмов по времени и ускорению в зависимости от числа потоков и размера массива.
\end{enumerate}

\section*{Технические характеристики}

Ниже приведены технические характеристики устройства, на котором были проведены эксперименты при помощи разработанного ПО:

\begin{itemize}
	\item операционная система: Windows 10 (64-разрядная);
	\item оперативная память: 32 GB;
	\item процессор: Intel(R) Core(TM) i7-7700K CPU @ 4.20GHz;
	\item количество ядер: 4;
	\item количество потоков: 8.
\end{itemize}

\section*{Постановка экспериментов}

Для измерения времени выполнения реализованных алгоритмов использовалась функция omp\_get\_wtime(), т.к. параллелизация производилась средствами технологии openMP. Для более точной оценки замеры проводились 100-1000 итераций в зависимости от размера массивов, а результат усреднялся.

%\newpage
\section*{Реализация на C}

Сначала был реализован последовательный алгоритм сортировки Хоара (листинг \ref{QuickSortSeq.cpp}). В познавательных целях были опробованы разбиение Ломуто и схема Хоара. После этого был реализован параллельный алгоритм Хоара с использованием технологии openMP (листинг \ref{QuickSortPar.cpp}). Была использована стратегия "Разделяй и властвуй", которая отлично ложится на алгоритм быстрой сортировки - при каждом новом рекурсивном вызове процесс разделяется на два и так, до тех пор, пока не будут использовано максимально возможно число потоков. Были опробованы варианты с omp parallel sections, omp task с условием if и просто omp task. Лучше всего показала себя последняя реализация, поэтому для сравнения с последовательным алгоритм использовалась именно она.

При первоначальной тестировании последовательного и параллельного алгоритмов получались неудовлетворительные результаты: параллельная реализация показывала себя хуже последовательной и не ускорялась с увеличением числа потоков. Но после включения оптимизации (флагом /O2 в Visual Studio) получились более корректные данные (рис. \ref{C_seq_par_compare}-\ref{C_seq_par_boost}).

\begin{figure}[!h]
	\begin{tikzpicture}
		\begin{axis}[
			xlabel={Кол-во потоков},
			ylabel={Время выполнения, сек.},
			xtick={1,2,4,8},
			legend pos=north east,
			ymajorgrids=true,
			grid style=dashed,
			width = 400
			]
			
			\addplot[
			color=blue,
			mark=square,
			]
			coordinates {
				(1, 0.002042)(2, 0.001761)(4, 0.001609)(8, 0.001346)
			};
			\addlegendentry{N = 25000}
			
			\addplot[
			color=green,
			mark=square,
			]
			coordinates {
				(1, 0.004184)(2, 0.003417)(4, 0.003136)(8, 0.002671)
			};
			\addlegendentry{N = 50000}
			
			\addplot[
			color=red,
			mark=square,
			]
			coordinates {
				(1, 0.008108)(2, 0.006589)(4, 0.005979)(8, 0.005384)
			};
			\addlegendentry{N = 100000}
			
		\end{axis}
	\end{tikzpicture}
	\caption{Зависимость времени выполнения сортировки от количества процессов для массивов различного размера}
	\label{C_seq_par_compare}
\end{figure}

\newpage
\begin{figure}[!h]
	\begin{tikzpicture}
		\begin{axis}[
			xlabel={Кол-во потоков},
			ylabel={Ускорение},
			xtick={1,2,4,8},
			legend pos=north west,
			ymajorgrids=true,
			grid style=dashed,
			width = 400
			]
			
			\addplot[
			color=blue,
			mark=square,
			]
			coordinates {
				(1, 0.002042/0.002042)(2, 0.002042/0.001761)(4, 0.002042/0.001609)(8, 0.002042/0.001346)
			};
			\addlegendentry{N = 25000}
			
			\addplot[
			color=green,
			mark=square,
			]
			coordinates {
				(1, 0.004184/0.004184)(2, 0.004184/0.003417)(4, 0.004184/0.003136)(8, 0.004184/0.002671)
			};
			\addlegendentry{N = 50000}
			
			\addplot[
			color=red,
			mark=square,
			]
			coordinates {
				(1, 0.008108/0.008108)(2, 0.008108/0.006589)(4, 0.008108/0.005979)(8, 0.008108/0.005384)
			};
			\addlegendentry{N = 100000}
			
		\end{axis}
	\end{tikzpicture}
	\caption{Зависимость ускорения сортировки от количества процессов для массивов различного размера}
	\label{C_seq_par_boost}
\end{figure}

Как видно из графиков \ref{C_seq_par_compare}-\ref{C_seq_par_boost} параллельная реализация работает быстрее последовательной, а ускорение для массивов различного размера примерно одинаковое, но не превышает 1.6.

Дополнительно два данных алгоритма были протестированы на суперкомпьютере Харизма (рис. \ref{C_seq_par_compare2}-\ref{C_seq_par_boost2}).

\newpage
\begin{figure}[!h]
	\begin{tikzpicture}
		\begin{axis}[
			xlabel={Кол-во потоков},
			ylabel={Время выполнения, сек.},
			xtick={1,2,4,8,16},
			legend pos=north west,
			ymajorgrids=true,
			grid style=dashed,
			width = 400
			]
			
			\addplot[
			color=blue,
			mark=square,
			]
			coordinates {
				(1, 0.001119)(2, 0.000879)(4, 0.000720)(8, 0.000626)(16, 0.006489)
			};
			\addlegendentry{N = 25000}
			
			\addplot[
			color=green,
			mark=square,
			]
			coordinates {
				(1, 0.002256)(2, 0.001774)(4, 0.001379)(8, 0.001252)(16, 0.002991)
			};
			\addlegendentry{N = 50000}
			
			\addplot[
			color=red,
			mark=square,
			]
			coordinates {
				(1, 0.004558)(2, 0.003433)(4, 0.003178)(8, 0.002465)(16, 0.002045)
			};
			\addlegendentry{N = 100000}
			
		\end{axis}
	\end{tikzpicture}
	\caption{Зависимость времени выполнения сортировки от количества процессов для массивов различного размера на Харизме}
	\label{C_seq_par_compare2}
\end{figure}

\newpage
\begin{figure}[!h]
	\begin{tikzpicture}
		\begin{axis}[
			xlabel={Кол-во потоков},
			ylabel={Ускорение},
			xtick={1,2,4,8,16},
			legend pos=north west,
			ymajorgrids=true,
			grid style=dashed,
			width = 400
			]
			
			\addplot[
			color=blue,
			mark=square,
			]
			coordinates {
				(1, 0.001119/0.001119)(2, 0.001119/0.000879)(4, 0.001119/0.000720)(8, 0.001119/0.000626)(16, 0.001119/0.006489)
			};
			\addlegendentry{N = 25000}
			
			\addplot[
			color=green,
			mark=square,
			]
			coordinates {
				(1, 0.002256/0.002256)(2, 0.002256/0.001774)(4, 0.002256/0.001379)(8, 0.002256/0.001252)(16, 0.002256/0.002991)
			};
			\addlegendentry{N = 50000}
			
			\addplot[
			color=red,
			mark=square,
			]
			coordinates {
				(1, 0.004558/0.004558)(2, 0.004558/0.003433)(4, 0.004558/0.003178)(8, 0.004558/0.002465)(16, 0.004558/0.002045)
			};
			\addlegendentry{N = 100000}
			
		\end{axis}
	\end{tikzpicture}
	\caption{Зависимость ускорения сортировки от количества процессов для массивов различного размера на Харизме}
	\label{C_seq_par_boost2}
\end{figure}

По графикам \ref{C_seq_par_compare2}-\ref{C_seq_par_boost2} видно, что результат аналогичен, но быстрее чем на предыдущей машине и при меньших размерах массивов с 16 потоками наблюдается ухудшение результатов по сравнению с последовательным вычислением. Также стоит заметить, что для наибольшего массива ускорение особенно заметно.

\section*{Реализация на python}

На питоне аналогично были реализованы последовательные (листинг \ref{QuickSortSeq.py}) и параллельные (листинги \ref{QuickSortPar1.py}-\ref{QuickSortPar2.py}) алгоритмы сортировки Хоара.

Первая последовательная реализация не использует декоратор @njit, а поэтому является самой медленной и показала результат хуже чем аналогичный алгоритм на C примерно в 5 раз. Вторая и третья реализации используют декоратор @njit, но одна из них рекурсивная, а другая итерационная и использует стек и поэтому медленней и для сравнения с параллельными алгоритмами использовалась рекурсивная реализация с декоратором @njit.

Одна параллельная реализация сделана при помощи библиотеки numba и декоратора @njit(parallel=True), а вторая при помощи numba.openmp. Библиотеку numba.openmp возможно использовать только на Linux подобных системах, поэтому она была установлена в docker контейнер, но в нем было доступно использование только четырех потоков. Для чистоты эксперимента, при сравнении алгоритмов на Python, все вычисления производились через docker контейнер.

На графиках \ref{Python_seq_par_compare1}-\ref{Python_seq_par_boost2} представлены результаты оценки времени работы данных алгоритмов. На первых двух графиков время выполнения, а на следующих двух ускорение.

\begin{figure}[!h]
	\begin{tikzpicture}
		\begin{axis}[
			xlabel={Кол-во потоков},
			ylabel={Время выполнения, сек.},
			xtick={1,2,4,8},
			legend pos=north east,
			ymajorgrids=true,
			grid style=dashed,
			width = 400
			]
			
			\addplot[
			color=blue,
			mark=square,
			]
			coordinates {
				(1, 0.009748058319091797)(2, 0.007573881149291992)(4, 0.009216516017913819)
			};
			\addlegendentry{N = 250000}
			
			\addplot[
			color=green,
			mark=square,
			]
			coordinates {
				(1, 0.01955103874206543)(2, 0.01593193769454956)(4, 0.019347262382507325 )
			};
			\addlegendentry{N = 500000}
			
			\addplot[
			color=red,
			mark=square,
			]
			coordinates {
				(1, 0.039735116958618165)(2, 0.0320928955078125)(4, 0.037515161037445066 )
			};
			\addlegendentry{N = 1000000}
			
		\end{axis}
	\end{tikzpicture}
	\caption{Зависимость времени выполнения сортировки от количества процессов для массивов различного размера с использованием  numba}
	\label{Python_seq_par_compare1}
\end{figure}

\newpage
\begin{figure}[!h]
\begin{tikzpicture}
	\begin{axis}[
		xlabel={Кол-во потоков},
		ylabel={Время выполнения, сек.},
		xtick={1,2,4,8},
		legend pos=north west,
		ymajorgrids=true,
		grid style=dashed,
		width = 400
		]
		
		\addplot[
		color=blue,
		mark=square,
		]
		coordinates {
			(1, 0.009748058319091797)(2, 0.011055986881256103)(4, 0.010998790264129638)
		};
		\addlegendentry{N = 250000}
		
		\addplot[
		color=green,
		mark=square,
		]
		coordinates {
			(1, 0.01955103874206543)(2, 0.02233180046081543)(4, 0.022407388687133788)
		};
		\addlegendentry{N = 500000}
		
		\addplot[
		color=red,
		mark=square,
		]
		coordinates {
			(1, 0.039735116958618165)(2, 0.046410486698150635)(4, 0.04675705909729004)
		};
		\addlegendentry{N = 1000000}
		
	\end{axis}
\end{tikzpicture}
\caption{Зависимость времени выполнения сортировки от количества процессов для массивов различного размера с использованием  numba.openmp}
\label{Python_seq_par_compare2}
\end{figure}

\newpage
\begin{figure}[!h]
\begin{tikzpicture}
	\begin{axis}[
		xlabel={Кол-во потоков},
		ylabel={Ускорение},
		xtick={1,2,4,8},
		legend pos=north east,
		ymajorgrids=true,
		grid style=dashed,
		width = 400
		]
		
		\addplot[
		color=blue,
		mark=square,
		]
		coordinates {
			(1, 0.009748058319091797/0.009748058319091797)(2, 0.009748058319091797/0.007573881149291992)(4, 0.009748058319091797/0.009216516017913819)
		};
		\addlegendentry{N = 250000}
		
		\addplot[
		color=green,
		mark=square,
		]
		coordinates {
			(1, 0.01955103874206543/0.01955103874206543)(2, 0.01955103874206543/0.01593193769454956)(4, 0.01955103874206543/0.019347262382507325 )
		};
		\addlegendentry{N = 500000}
		
		\addplot[
		color=red,
		mark=square,
		]
		coordinates {
			(1, 0.039735116958618165/0.039735116958618165)(2, 0.039735116958618165/0.0320928955078125)(4, 0.039735116958618165/0.037515161037445066 )
		};
		\addlegendentry{N = 1000000}
		
	\end{axis}
\end{tikzpicture}
\caption{Зависимость ускорения сортировки от количества процессов для массивов различного размера с использованием  numba}
\label{Python_seq_par_boost1}
\end{figure}

\newpage
\begin{figure}[!h]
\begin{tikzpicture}
	\begin{axis}[
		xlabel={Кол-во потоков},
		ylabel={Ускорение},
		xtick={1,2,4,8},
		legend pos=north west,
		ymajorgrids=true,
		grid style=dashed,
		width = 400
		]
		
		\addplot[
		color=blue,
		mark=square,
		]
		coordinates {
			(1, 0.009748058319091797/0.009748058319091797)(2, 0.009748058319091797/0.011055986881256103)(4, 0.009748058319091797/0.010998790264129638)
		};
		\addlegendentry{N = 250000}
		
		\addplot[
		color=green,
		mark=square,
		]
		coordinates {
			(1, 0.01955103874206543/0.01955103874206543)(2, 0.01955103874206543/0.02233180046081543)(4, 0.01955103874206543/0.022407388687133788)
		};
		\addlegendentry{N = 500000}
		
		\addplot[
		color=red,
		mark=square,
		]
		coordinates {
			(1, 0.039735116958618165/0.039735116958618165)(2, 0.039735116958618165/0.046410486698150635)(4, 0.039735116958618165/0.04675705909729004)
		};
		\addlegendentry{N = 1000000}
		
	\end{axis}
\end{tikzpicture}
\caption{Зависимость ускорения сортировки от количества процессов для массивов различного размера с использованием  numba.openmp}
\label{Python_seq_par_boost2}
\end{figure}

По графику \ref{Python_seq_par_compare1} и \ref{Python_seq_par_boost1} видно, что заметное ускорение происходит только при двух потоках. А в случае использования numba.openmp, ускорения вовсе не наблюдается, более того данная реализация показывает себя медленнее.

\section*{Сравнение C и Python}

Т.к. реализация на Python с использованием numba показала себя лучше, будем использовать ее для сравнения. 
Для начала приведем графики для массивов одинакового размера (N = 100000) на обоих языках (\ref{Python_C_compare}-\ref{Python_C_boost})

\newpage
\begin{figure}[!h]
	\begin{tikzpicture}
		\begin{axis}[
			xlabel={Кол-во потоков},
			ylabel={Время выполнения, сек.},
			xtick={1,2,4,8},
			legend pos=north east,
			ymajorgrids=true,
			grid style=dashed,
			width = 400
			]
			
			\addplot[
			color=blue,
			mark=square,
			]
			coordinates {
				(1, 0.008108)(2, 0.006589)(4, 0.005979)
			};
			\addlegendentry{C}
			
			\addplot[
			color=red,
			mark=square,
			]
			coordinates {
				(1, 0.0037239837646484376)(2, 0.0030921816825866697)(4, 0.0036490726470947265 )
			};
			\addlegendentry{Python}
			
		\end{axis}
	\end{tikzpicture}
	\caption{Зависимость времени выполнения сортировки от количества процессов для }
	\label{Python_C_compare}
\end{figure}


\begin{figure}[!h]
	\begin{tikzpicture}
		\begin{axis}[
			xlabel={Кол-во потоков},
			ylabel={Ускорение},
			xtick={1,2,4,8},
			legend pos=north east,
			ymajorgrids=true,
			grid style=dashed,
			width = 400
			]
			
			\addplot[
			color=blue,
			mark=square,
			]
			coordinates {
				(1, 0.008108/0.008108)(2, 0.008108/0.006589)(4, 0.008108/0.005979)
			};
			\addlegendentry{C}
			
			\addplot[
			color=red,
			mark=square,
			]
			coordinates {
				(1, 0.0037239837646484376/0.0037239837646484376)(2, 0.0037239837646484376/0.0030921816825866697)(4, 0.0037239837646484376/0.0036490726470947265 )
			};
			\addlegendentry{Python}
			
		\end{axis}
	\end{tikzpicture}
	\caption{Ускорение сортировки от количества процессов для }
	\label{Python_C_boost}
\end{figure}

\newpage
Из графиков \ref{Python_C_compare}-\ref{Python_C_boost} можно сделать вывод, что реализация на python быстрее из-за оптимизации numba при помощи декоратора @njit, т.к. без него алгоритм работает медленнее чем на C. Но при этом ускорение на языке C увеличивается до 8 потоков, а на Python только до 2.

\section*{Выводы}
По итогу проделанной работы были реализованы последовательные и параллельные алгоритмы сортировки Хоара на языках C и Python с использованием технологии OpenMP, а также проведено их исследование.

Из экспериментов можно сделать вывод, что очень важно оптимально распараллеливать алгоритм для получения выигрыша по времени, а также большую роль в этом играет оптимизация кода. 

\chapter*{Листинги}

\begin{lstinputlisting}[
	caption={Последовательный алгоритм сортировки Хоара на языке C},
	label={QuickSortSeq.cpp},
	style={c},
	]{src/QuickSortSeq.cpp}
\end{lstinputlisting}

\begin{lstinputlisting}[
caption={Параллельные алгоритмы сортировки Хоара на языке C},
label={QuickSortPar.cpp},
style={c},
]{src/QuickSortPar.cpp}
\end{lstinputlisting}

\newpage
\begin{lstinputlisting}[
	caption={Последовательные алгоритм сортировки Хоара на языке Python},
	label={QuickSortSeq.py},
	style={c},
	language={Python}
	]{src/QuickSortSeq.py}
\end{lstinputlisting}

\newpage
\begin{lstinputlisting}[
caption={Параллельный алгоритм сортировки Хоара на языке Python с использованием numba},
label={QuickSortPar1.py},
style={c},
language={Python}
]{src/QuickSortPar1.py}
\end{lstinputlisting}

\newpage
\begin{lstinputlisting}[
caption={Параллельный алгоритм сортировки Хоара на языке Python с использованием numba.openmp},
label={QuickSortPar2.py},
style={c},
language={Python}
]{src/QuickSortPar2.py}
\end{lstinputlisting}

\end{document}
\documentclass[a4paper,oneside,14pt]{extreport}

\include{preamble}

\begin{document}
	
% ЗАДАНИЕ 2a Pinned Memory
\begin{figure}
	\begin{tikzpicture}
		\begin{axis}[
			xlabel={Размер матриц},
			ylabel={Время копирования, сек.},
			xtick={1000,2000,3000,4000,5000},
			legend pos=north west,
			ymajorgrids=true,
			grid style=dashed,
			width = 400
			]
			
			\addplot[
			color=blue,
			mark=square,
			]
			coordinates {
				(1000, 0.000497)(2000, 0.001767)(3000, 0.004237)(4000, 0.005760)(5000, 0.011798)
			};
			\addlegendentry{Pinned Memory}
			
			
			\addplot[
			color=green,
			mark=square,
			]
			coordinates {
				(1000, 0.000906)(2000, 0.003129)(3000, 0.007427)(4000, 0.013596)(5000, 0.021703)
			};
			\addlegendentry{Global Memory}
			
		\end{axis}
	\end{tikzpicture}
	\caption{Задание 2a - Зависимость времени копирования матриц от размера}
\end{figure}

% ЗАДАНИЕ 2d Shared Memory
\begin{figure}
	\begin{tikzpicture}
		\begin{axis}[
			xlabel={Размер матриц},
			ylabel={Время выполнения, сек.},
			xtick={500,1000,1500},
			legend pos=north west,
			ymajorgrids=true,
			grid style=dashed,
			width = 400
			]
			
			\addplot[
			color=blue,
			mark=square,
			]
			coordinates {
				(500, 0.000018)(1000, 0.000032)(1500, 0.000048)
			};
			\addlegendentry{Shared Memory}
			
			
			\addplot[
			color=green,
			mark=square,
			]
			coordinates {
				(500, 0.000020)(1000, 0.000037)(1500, 0.000055)
			};
			\addlegendentry{Global Memory}
			
		\end{axis}
	\end{tikzpicture}
	\caption{Задание 2d - Зависимость времени выполнения умножения матриц от размера матриц и типа памяти}
\end{figure}

\begin{figure}
	\begin{tikzpicture}
		\begin{axis}[
			xlabel={Размер матриц},
			ylabel={GFLOPS},
			xtick={500,1000,1500},
			legend pos=north west,
			ymajorgrids=true,
			grid style=dashed,
			width = 400
			]
			
			\addplot[
			color=blue,
			mark=square,
			]
			coordinates {
				(500, 14209.971018)(1000, 61788.198219)(1500, 140616.557136)
			};
			\addlegendentry{Shared Memory}
			
			
			\addplot[
			color=green,
			mark=square,
			]
			coordinates {
				(500, 12809.896709)(1000, 54658.187920)(1500, 123777.266996)
			};
			\addlegendentry{Global Memory}
			
		\end{axis}
	\end{tikzpicture}
	\caption{Задание 2d - Зависимость GFLOPS умножения матриц от размера матриц и типа памяти}
\end{figure}

% ЗАДАНИЕ 4-5 cuBlas openMP
\begin{figure}
	\begin{tikzpicture}
		\begin{axis}[
			xlabel={Размер матриц},
			ylabel={Время выполнения, сек.},
			xtick={500,1000,1500},
			legend pos=north west,
			ymajorgrids=true,
			grid style=dashed,
			width = 400
			]
			
			\addplot[
			color=blue,
			mark=square,
			]
			coordinates {
				(500, 0.000038)(1000, 0.000080)(1500, 0.000122)
			};
			\addlegendentry{Моя реализация}
			
			
			\addplot[
			color=green,
			mark=square,
			]
			coordinates {
				(500, 0.000029)(1000, 0.000171)(1500, 0.000556)
			};
			\addlegendentry{Библиотека CuBlas}
			
			\addplot[
			color=red,
			mark=square,
			]
			coordinates {
				(500, 0.001916)(1000, 0.011163)(1500, 0.041715)
			};
			\addlegendentry{OpenMP}
			
		\end{axis}
	\end{tikzpicture}
	\caption{Задания 4 и 5 - Зависимость времени выполнения умножения матриц от размера матриц и используемой библиотеки}
\end{figure}

\begin{figure}
	\begin{tikzpicture}
		\begin{axis}[
			xlabel={Размер матриц},
			ylabel={GFLOPS},
			xtick={500,1000,1500},
			legend pos=north west,
			ymajorgrids=true,
			grid style=dashed,
			width = 400
			]
			
			\addplot[
			color=blue,
			mark=square,
			]
			coordinates {
				(500, 6513.912684)(1000, 25041.407868)(1500, 55520.037230)
			};
			\addlegendentry{Моя реализация}
			
			
			\addplot[
			color=green,
			mark=square,
			]
			coordinates {
				(500, 8723.436088)(1000, 11675.251428)(1500, 12138.911241)
			};
			\addlegendentry{Библиотека CuBlas}
			
			\addplot[
			color=red,
			mark=square,
			]
			coordinates {
				(500, 130.473453)(1000, 179.160306)(1500, 161.811457)
			};
			\addlegendentry{OpenMP}
			
		\end{axis}
	\end{tikzpicture}
	\caption{Задания 4 и 5 - Зависимость GFLOPS умножения матриц от размера матриц и используемой библиотеки}
\end{figure}

\end{document}
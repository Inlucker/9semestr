\documentclass[a4paper,oneside,14pt]{extreport}

\include{preamble}

\begin{document}

\chapter*{Лабораторная работа 1}
Цель: изучение дискретного преобразования Фурье в Matlab.
\section*{Задание 1 – анализ сгенерированного сигнала}
Сгенерировать сигнал продолжительностью 90 лет с шагом месяц => всего будет $90*12=1080$ шагов. Сигнал содержит косинус периодом 1 год, синус периодом 8.86 года и еще один косинус периодом 18.6 года => период 1-ой функции $12$ месяцев, 2-ой $8.86*12=106.32$ месяцев и 3-ей функции $12*18.6=223.2$ месяцев. Начало - дата рождения (05.2000) => $2000*12+5=24005$ => начинаем с 24005-го месяца. Подбираем фазы так, чтобы нулевой аргумент годовой гармоники приходился на начало года,  8.86-летнего цикла – на начало 2024 года, 18.6-летнего цикла на начало 2006 года => $\phi_1=0, \phi_2=1.12*\pi, \phi_3=0.31*\pi$. Амплитуды сигналов выбрать в соответствии с правилом: 20+ номер по алфавиту первых букв Ваших ФИО => Пронин Арсений Сергеевич (ПАС - номера букв в алфавите = 16, 0, 18) $A_1=36, A_2=20, A_3=38$.

Выполнить спектральный анализ сигнала, построить график амплитудного спектра (рис. \ref{task1_ampl_spectr}):
\newline
\begin{figure}[h]
	\center{\includegraphics[width=1\linewidth]{inc/task1_ampl_spectr}}
	\caption{График амплитудного спектра}
	\label{task1_ampl_spectr}
\end{figure}

\begin{figure}[h]
	\center{\includegraphics[width=1\linewidth]{inc/task1_ampl_spectr_vals}}
	\caption{Пиковые значения}
	\label{task1_ampl_spectr_vals}
\end{figure}

На рисунке \ref{task1_ampl_spectr_vals} видно что пиковые значения Y близки к амплитудам, а соответствующие им X $\approx$ циклическим частотам [$\frac{\text{радиан}}{\text{месяц}}$].

\section*{Задание 2 – фурье-анализ реального сигнала}
Считать из бюллетеня EOP C01 службы вращения Земли данные UT-TAI (6-я колонка) и построить график (рис. \ref{task2_g1}):

\begin{figure}[h]
	\center{\includegraphics[width=1\linewidth]{inc/task2_g1}}
	\caption{Зависимость UTTAI от YEARS}
	\label{task2_g1}
\end{figure}

\newpage
Продифференцировать данные, получив LOD (минус производная, рис. \ref{task2_g2}):

\begin{figure}[h]
	\center{\includegraphics[width=0.93\linewidth]{inc/task2_g2}}
	\caption{Зависимость LOD от YEARS c 1962-го года}
	\label{task2_g2}
\end{figure}

Взять ряд с 1962 года  и вычислить его спектр. Построить график амплитудного спектра по частотам (рис. \ref{task2_g3}):

\begin{figure}[h]
	\center{\includegraphics[width=0.93\linewidth]{inc/task2_g3}}
	\caption{График амплитудного спектра по частотам LOD c 1962-го года}
	\label{task2_g3}
\end{figure}

\newpage
По пиковым значениям (рис. \ref{task2_freq_vals}) можно определить циклические частоты: 

$0.00511244 [\frac{\text{радиан}}{\frac{1}{20}\text{года}}]$

$0.311859 [\frac{\text{радиан}}{\frac{1}{20}\text{года}}]$

$0.62883 [\frac{\text{радиан}}{\frac{1}{20}\text{года}}]$

%и Периоды:
%
%$\Large \frac{2 \cdot \pi}{0.00511244} \approx 1229$ лет
%
%$\Large \frac{2 \cdot \pi}{0.311859} \approx 20$ лет
%
%$\Large \frac{2 \cdot \pi}{0.62883} \approx 10$ лет

\begin{figure}[h]
	\center{\includegraphics[width=1\linewidth]{inc/task2_freq_vals}}
	\caption{Значения частот}
	\label{task2_freq_vals}
\end{figure}

%\newpage
\section*{Задание 3 – построение спектра комплексного ряда}

Из того же файла считать вторую (x) и четвертую (y) колонки и объединить их в  комплексный временной ряд (x+iy). Построить график спектра как для положительных так и для отрицательных частот:

\begin{figure}[h]
	\center{\includegraphics[width=1\linewidth]{inc/task3_g1}}
	\caption{График спектра}
	\label{task3_g1}
\end{figure}

\end{document}
\documentclass[a4paper,oneside,14pt]{extreport}

\include{preamble}

\begin{document}

\begin{titlepage}
	\centering
	
	\vspace{-2.2mm}
	\vhrulefill{0.9mm}\\
	\vspace{-7mm}
	\vhrulefill{0.2mm}\\
	\vspace{2mm}
	
	\vspace{50mm}
	
	\vspace{30mm}
	
	\textbf{Отчет по лабораторной работе №4}\\
	По курсу: <<Фильтрация и прогнозирование данных>>\\
	Тема: <<Сингулярный спектральный анализ>>\\
	
	\vspace{60mm}
	
	\hspace{70mm} Студент:       \hfill Пронин~А.~С.\\
	\hspace{70mm} Группа:        \hfill МСМТ231\\
	\hspace{70mm} Преподаватель: \hfill Зотов~Л.~В.\\
	%	\hspace{70mm} Оценка:        \hfill \hrulefill\\
	
	\vfill
	
	Москва\\
	\the\year
\end{titlepage}

\setcounter{page}{2}

\chapter*{Лабораторная работа 3}

Задание 1 – Добавить к сигналу из ЛР N1 авторегрессионный шум

Исходный сигнал из ЛР 1 (рис. \ref{lr1_signal}):

\begin{figure}[!h]
	\center{\includegraphics[width=1\linewidth]{inc/lr1_signal}}
	\caption{Исходный сигнал из ЛР 1}
	\label{lr1_signal}
\end{figure}

\newpage
Добавим к нему авторегрессионный шум (рис. \ref{task1_shum1}-\ref{task1_signal_shum1}):

\begin{figure}[!h]
	\center{\includegraphics[width=1\linewidth]{inc/task1_shum1}}
	\caption{Шум 1}
	\label{task1_shum1}
\end{figure}

\begin{figure}[!h]
	\center{\includegraphics[width=1\linewidth]{inc/task1_signal_shum1}}
	\caption{Сигнал с шумом 1}
	\label{task1_signal_shum1}
\end{figure}

\newpage
По графикам \ref{task1_shum1}-\ref{task1_signal_shum1} видно, что сигнал почти не изменился, поэтому увеличим разброс шума (рис. \ref{task1_shum2}-\ref{task1_signal_shum2}):

\begin{figure}[!h]
	\center{\includegraphics[width=1\linewidth]{inc/task1_shum2}}
	\caption{Шум 2}
	\label{task1_shum2}
\end{figure}

\begin{figure}[!h]
	\center{\includegraphics[width=1\linewidth]{inc/task1_signal_shum2}}
	\caption{Сигнал с шумом 2}
	\label{task1_signal_shum2}
\end{figure}

\newpage
По заданию 2 сохраним полученный сигнал в файл signal.mat и обработаем его с помощью Wavelet Analyzer (рис. \ref{task5}):

\begin{figure}[!h]
	\center{\includegraphics[width=1\linewidth]{inc/task5}}
	\caption{Wavelet Analyzer}
	\label{task5}
\end{figure}

Используем старую версию cwt для нашего сигнала (рис. \ref{task2}):

\begin{figure}[!h]
	\center{\includegraphics[width=1\linewidth]{inc/task2}}
	\caption{Старая версия cwt}
	\label{task2}
\end{figure}

\newpage
На рисунке \ref{task2} можно увидеть что в результате Вейвлет анализа отобразилась информация как о низких, так и о высоких частотах. Изучив коэффициенты вейвлет-преобразования, можно выделить частотные компоненты сигнала. Это позволяет локализовать и анализировать различные частоты в сигнале.

По заданию 4 построим 3D поверхность для данной скалограммы (рис. \ref{task4}):

\begin{figure}[!h]
	\center{\includegraphics[width=1\linewidth]{inc/task4}}
	\caption{3D поверхность}
	\label{task4}
\end{figure}

\newpage
По заданию 6 добавим импульс в сигнал для сотого элемента и изучим треугольник его влияния (рис. \ref{task6}):

\begin{figure}[!h]
	\center{\includegraphics[width=1\linewidth]{inc/task6}}
	\caption{Вейвлет анализ сигнала с импульсом}
	\label{task6}
\end{figure}

Как видно по рисунку \ref{task6}, как и ожидалось, треугольник исходящий из сотого элемента выделяется среди других. Из этого можно сделать вывод что Вейвлет анализ также предоставляет информацию о локальных изменениях в сигнале, что помогает выявлять переходы между различными участками сигнала и определять их структуру.

\newpage
По заданию 7 выполним Вейвлет анализ сигнала LOD из файла eopc\_C02.dat (рис. \ref{task7}):

\begin{figure}[!h]
	\center{\includegraphics[width=1\linewidth]{inc/task7}}
	\caption{Вейвлет анализ сигнала LOD из файла eopc\_C02.dat}
	\label{task7}
\end{figure}

\end{document}